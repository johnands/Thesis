\documentclass{beamer}

\usepackage[utf8]{inputenc}
\usepackage{default}
\usepackage{amsmath}

\title{Konstruksjon av høydimensjonale nevralt nettverk-potensialer for molekylærdynamikk}
\author{John-Anders Stende}
\institute{
  Fysisk institutt \\
  Universitetet i Oslo
}
\date{Masterpresentasjon, oktober 2017}
\subject{Computational physics}

\begin{document}

\frame{\titlepage}


\begin{frame}

\begin{block}{Hva er molekylærdynamikk?}
  \begin{itemize}
  \item Numerisk metode for å simulere atomers og molekylers bevegelser i gasser, væsker og faste stoffer.
  \item Virtuelt eksperiment.
  \end{itemize}
\end{block}

\end{frame}


\begin{frame}

\begin{block}{Dynamikk}
  \begin{itemize}
  \item Partiklenes interaksjoner styrer dynamikken.
  \item Interaksjonene bestemmes av et kraftfelt $\mathbf{F}$:
    \begin{equation*}
    \mathbf{F} = -\nabla V(\mathbf{r})
    \end{equation*}
  \item Potensiell energiflate / potensial):
    \begin{equation*}
      V(\mathbf{r}), \quad \mathbf{r} = (\mathbf{r}_1, \mathbf{r}_2, \cdots, \mathbf{r}_N)
    \end{equation*}
  \item $V(\mathbf{r})$ inneholder all fysikken. 
  \end{itemize}
\end{block}

\end{frame}


\begin{frame}

\begin{columns}[T] % contents are top vertically aligned
  \begin{column}[T]{0.5\linewidth} % each column can also be its own environment
    \begin{block}{Ab inito molekylærdynamikk}
    Løse Schrödinger-likningen ved hvert tidssteg.
    \end{block}
  \end{column}
  \begin{column}[T]{0.5\linewidth} % alternative top-align that's better for graphics
    \begin{block}{Klassisk molekylærdynamikk}
    Bruke en predefinert analytisk funksjon.
    \end{block}
  \end{column}
\end{columns}

\end{frame}


\begin{frame}

\begin{block}{Klassisk potensial}
  \begin{equation*}
  V(\mathbf{r}) \approx \sum_i^N V_1(\mathbf{r}_i) + \sum_{i,j}^N V_2(\mathbf{r}_i, \mathbf{r}_j) + 
  \sum_{i,j,k}^N V_3(\mathbf{r}_i, \mathbf{r}_j, \mathbf{r}_k) + \dots
  \end{equation*}
\end{block}

\begin{block}{Hvordan bør leddene se ut?}
Eksperiementer / kvantemekanikk
\end{block}

\end{frame}


\begin{frame}

\begin{block}{''Fysisk'' strategi:}
 \begin{enumerate}
  \item Starte med en funksjonsform med noen parametre.
  \item Bestemme parametre fra eksperimentelle data.
 \end{enumerate}
\end{block}

\begin{block}{''Matematisk'' strategi:}
 \begin{enumerate}
  \item Produsere kvantemekaniske data. 
  \item Tilpasse en generell funksjonsform til dataene.
 \end{enumerate}
\end{block}

\end{frame}


\begin{frame}
 
\begin{block}{Interpolere datasett}
 \begin{itemize}
  \item Spliner
  \item Minste kvadraters metode
  \item Kunstige nevrale nettverk
 \end{itemize}
\end{block}
 
\end{frame}


\begin{frame}
 
\begin{block}{Kunstige nevrale nettverk}
 
\end{block}

 
\end{frame}








\end{document}
